%%%%%%%%%%%%%%%%%%%%%%%%%%%%%%%%%%%%%%%%%
% University/School Laboratory Report
% LaTeX Template
% Version 3.1 (25/3/14)
%
% This template has been downloaded from:
% http://www.LaTeXTemplates.com
%
% Original author:
% Linux and Unix Users Group at Virginia Tech Wiki 
% (https://vtluug.org/wiki/Example_LaTeX_chem_lab_report)
%
% License:
% CC BY-NC-SA 3.0 (http://creativecommons.org/licenses/by-nc-sa/3.0/)
%
%%%%%%%%%%%%%%%%%%%%%%%%%%%%%%%%%%%%%%%%%

%----------------------------------------------------------------------------------------
%	PACKAGES AND DOCUMENT CONFIGURATIONS
%----------------------------------------------------------------------------------------

\documentclass{report}
\usepackage[utf8]{inputenc}
\usepackage{siunitx} % Provides the \SI{}{} and \si{} command for typesetting SI units
\usepackage{graphicx} % Required for the inclusion of images
\usepackage{caption}
\usepackage{natbib} % Required to change bibliography style to APA
\usepackage{amsmath} % Required for some math elements 
\usepackage{amssymb}
\usepackage{blindtext}
\usepackage{tikz}
\usetikzlibrary{matrix}
%\setlength\parindent{0pt} % Removes all indentation from paragraphs
\usepackage[framed]{matlab-prettifier}
\definecolor{mygreen}{RGB}{28,172,0} % color values Red, Green, Blue
\definecolor{mylilas}{RGB}{170,55,241}
\usepackage{listings}
\lstset{
	style=Matlab-editor,
	basicstyle         = \fontsize{8}{11}\ttfamily,
	numberstyle       =\fontsize{8}{11}\ttfamily,
	%backgroundcolor=\color{gray},
	%mlshowsectionrules = true,
	rangeprefix        = \%\ 
}

\renewcommand{\labelenumi}{\alph{enumi}.} % Make numbering in the enumerate environment by letter rather than number (e.g. section 6)

%\usepackage{times} % Uncomment to use the Times New Roman font
%----------------------------------------------------------------------------------------
%	TITLE PAGE
%----------------------------------------------------------------------------------------
% Title Page
\title{TIC}
\author{Mauricio Caceres}
\date{18 décembre 2016}

%----------------------------------------------------------------------------------------
%	DOCUMENT INFORMATION
%----------------------------------------------------------------------------------------

\begin{document}
\begin{titlepage}
	\centering
	\vfill
{\bfseries\huge Detection et Estimation}
	\vfill
	{\bfseries\LARGE
		TP:\\
		Detecttion
		\\
		\vskip2cm

		Master SISEA\\
	
	}
	\vfill
	18 décembre 2016
	\vfill
	{\large Mauricio Caceres } \hfill  {\large Pierre-Samuel Garreau-Hamard}
	\vfill
	{\large Enseignant : Di Ge }
	\vfill
	\includegraphics[width=9cm]{rennes} % also works with logo.pdf    
	\vfill
	\includegraphics[width=6cm]{enssat} % also works with logo.pdf
	\vfill
	\vfill
\end{titlepage}


%\begin{center}
%\begin{tabular}{l r}
%Date Performed: & January 1, 2012 \\ % Date the experiment was performed
%Partners: & James Smith \\ % Partner names
%& Mary Smith \\
%Instructor: & Professor Smith % Instructor/supervisor
%\end{tabular}
%\end{center}


% If you wish to include an abstract, uncomment the lines below
% \begin{abstract}
% Abstract text
% \end{abstract}

%----------------------------------------------------------------------------------------
%	SECTION 1
%----------------------------------------------------------------------------------------
\chapter{Introduction}
\section{Objectif}



\section{Write the log likelihoood funciton $L(A,\phi)$ for a given observation vector}
\label{estimationProblem}

We have the signal 
\begin{equation}\label{key}
x(n) = A cos (2\pi f_0 n + \phi) + w(n)
\end{equation}

The non deterministic part of the signal is the noise. A White Gaussian noise with variance $\sigma ^2$, wich is common in many fields. So the log-likelihood function is evaluated in function of this noise with the next expression.
\begin{equation}\label{key}
w(n) = x(n) - A cos (2\pi f_0 n + \phi)
\end{equation}
The WGN distributions are iid so 
\begin{equation}\label{key}
P(w(n)|A,\theta ) = \prod_{n=0}^{N-1} P_w(w(n))
\end{equation}

So the log-likelihood function $L(A,\phi)$ is 

\begin{equation}\label{key}
L(A,\phi) = log P(w(n)|A,\theta )
\end{equation}

\begin{equation}\label{key}
L(A,\phi) = log \prod_{n=0}^{N-1} P_w(w(n))
\end{equation}

\begin{equation}\label{key}
L(A,\phi) = \sum_{n=0}^{N-1} log P_w(w(n))
\end{equation}

\begin{equation}\label{key}
L(A,\phi) = \sum_{n=0}^{N-1} log(\frac{1}{\sigma \sqrt{2\pi}}\exp{[-1/2(\frac{x(n) - A cos (2\pi f_0 n + \phi)-\mu}{\sigma})^2]}
\end{equation}

\begin{equation}\label{key}
L(A,\phi) = \sum_{n=0}^{N-1} B + ({[-1/2(\frac{x(n) - A cos (2\pi f_0 n + \phi)-\mu}{\sigma})^2]}a
\end{equation}

\section{Show that the maximun likelihood estimators are the solution of the following equations}

\begin{equation}\label{key}
\frac{\partial L(A,\phi)}{\partial A} = \sum_{n=0}^{N-1} (\frac{x(n) - A cos (2\pi f_0 n + \phi)-\mu}{\sigma})(\frac{cos(2\pi f_0 n +\phi}{\sigma})
\end{equation}

\begin{equation}\label{key}
\frac{\partial L(A,\phi)}{\partial \phi} = \sum_{n=0}^{N-1} - (\frac{x(n) - A cos (2\pi f_0 n + \phi)-\mu}{\sigma})(\frac{Asin(2\pi f_0 n +\phi}{\sigma})
\end{equation}


\begin{equation}\label{key}
\frac{\partial L(A,\phi)}{\partial \phi} = 0 = \sum_{n=0}^{N-1} - (\frac{Ax(n)sin(2\phi f_0 n +\phi)}{\sigma^2}-\frac{A^2cos(2\phi f_0 n +\phi)sin(2\phi f_0 n +\phi)}{\sigma^2}) = 0
\end{equation}


\begin{equation}\label{key}
\frac{\partial L(A,\phi)}{\partial \phi} = 0 = \sum_{n=0}^{N-1} - (\frac{Ax(n)sin(2\phi f_0 n +\phi)}{\sigma^2}-\frac{A^2cos(2\phi f_0 n +\phi)sin(2\phi f_0 n +\phi)}{\sigma^2}) = 0
\end{equation}

\begin{equation}\label{key}
\frac{\partial L(A,\phi)}{\partial \phi} = 0 = \frac{A}{\sigma^2} \sum_{n=0}^{N-1} - (\frac{Ax(n)sin(2\phi f_0 n +\phi)-A^2cos(2\phi f_0 n +\phi)sin(2\phi f_0 n +\phi)}{\sigma^2}) = 0
\end{equation}












%----------------------------------------------------------------------------------------
%	SECTION 2
%----------------------------------------------------------------------------------------
\chapter{Implémentation en MATLAB{\small \circledR}}




\section{Estimation de l'information mutuelle}



%----------------------------------------------------------------------------------------
%	SECTION 3
%----------------------------------------------------------------------------------------


\section{Conclusion}


%----------------------------------------------------------------------------------------
%	FIN
%----------------------------------------------------------------------------------------



\end{document}























