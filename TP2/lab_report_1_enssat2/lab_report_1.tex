%%%%%%%%%%%%%%%%%%%%%%%%%%%%%%%%%%%%%%%%%
% University/School Laboratory Report
% LaTeX Template
% Version 3.1 (25/3/14)
%
% This template has been downloaded from:
% http://www.LaTeXTemplates.com
%
% Original author:
% Linux and Unix Users Group at Virginia Tech Wiki 
% (https://vtluug.org/wiki/Example_LaTeX_chem_lab_report)
%
% License:
% CC BY-NC-SA 3.0 (http://creativecommons.org/licenses/by-nc-sa/3.0/)
%
%%%%%%%%%%%%%%%%%%%%%%%%%%%%%%%%%%%%%%%%%

%----------------------------------------------------------------------------------------
%	PACKAGES AND DOCUMENT CONFIGURATIONS
%----------------------------------------------------------------------------------------

\documentclass{report}
\usepackage[utf8]{inputenc}
\usepackage{siunitx} % Provides the \SI{}{} and \si{} command for typesetting SI units
\usepackage{graphicx} % Required for the inclusion of images
\usepackage{caption}
\usepackage{natbib} % Required to change bibliography style to APA
\usepackage{amsmath} % Required for some math elements 
\usepackage{amssymb}
\usepackage{blindtext}
\usepackage{tikz}
\usetikzlibrary{matrix}
%\setlength\parindent{0pt} % Removes all indentation from paragraphs
\usepackage[framed]{matlab-prettifier}
\definecolor{mygreen}{RGB}{28,172,0} % color values Red, Green, Blue
\definecolor{mylilas}{RGB}{170,55,241}
\usepackage{listings}
\lstset{
	style=Matlab-editor,
	basicstyle         = \fontsize{8}{11}\ttfamily,
	numberstyle       =\fontsize{8}{11}\ttfamily,
	%backgroundcolor=\color{gray},
	%mlshowsectionrules = true,
	rangeprefix        = \%\ 
}

\renewcommand{\labelenumi}{\alph{enumi}.} % Make numbering in the enumerate environment by letter rather than number (e.g. section 6)

%\usepackage{times} % Uncomment to use the Times New Roman font
%----------------------------------------------------------------------------------------
%	TITLE PAGE
%----------------------------------------------------------------------------------------
% Title Page
\title{TIC}
\author{Mauricio Caceres}
\date{18 décembre 2016}

%----------------------------------------------------------------------------------------
%	DOCUMENT INFORMATION
%----------------------------------------------------------------------------------------

\begin{document}
\begin{titlepage}
	\centering
	\vfill
{\bfseries\huge Detection et Estimation}
	\vfill
	{\bfseries\LARGE
		TP:\\
		Detecttion
		\\
		\vskip2cm

		Master SISEA\\
	
	}
	\vfill
	18 décembre 2016
	\vfill
	{\large Mauricio Caceres } \hfill  {\large Pierre-Samuel Garreau-Hamard}
	\vfill
	{\large Enseignant : Di Ge }
	\vfill
	\includegraphics[width=9cm]{rennes} % also works with logo.pdf    
	\vfill
	\includegraphics[width=6cm]{enssat} % also works with logo.pdf
	\vfill
	\vfill
\end{titlepage}


%\begin{center}
%\begin{tabular}{l r}
%Date Performed: & January 1, 2012 \\ % Date the experiment was performed
%Partners: & James Smith \\ % Partner names
%& Mary Smith \\
%Instructor: & Professor Smith % Instructor/supervisor
%\end{tabular}
%\end{center}


% If you wish to include an abstract, uncomment the lines below
% \begin{abstract}
% Abstract text
% \end{abstract}

%----------------------------------------------------------------------------------------
%	SECTION 1
%----------------------------------------------------------------------------------------
\chapter{Introduction}
\section{Objectif}



\section{Write the log likelihoood funciton $L(A,\phi)$ for a given observation vector}
\label{estimationProblem}

We have the signal 
\begin{equation}\label{key}
x(n) = A cos (2\pi f_0 n + \phi) + w(n)
\end{equation}

The non deterministic part of the signal is the noise. A White Gaussian noise with variance $\sigma ^2$, wich is common in many fields. So the log-likelihood function is evaluated in function of this noise with the next expression.
\begin{equation}\label{key}
w(n) = x(n) - A cos (2\pi f_0 n + \phi)
\end{equation}
The WGN distributions are iid so 
\begin{equation}\label{key}
P(w(n)|A,\theta ) = \prod_{n=0}^{N-1} P_w(w(n))
\end{equation}

So the log-likelihood function $L(A,\phi)$ is 

\begin{equation}\label{key}
L(A,\phi) = log P(w(n)|A,\theta )
\end{equation}

\begin{equation}\label{key}
L(A,\phi) = log \prod_{n=0}^{N-1} P_w(w(n))
\end{equation}

\begin{equation}\label{key}
L(A,\phi) = \sum_{n=0}^{N-1} log P_w(w(n))
\end{equation}

\begin{equation}\label{key}
L(A,\phi) = \sum_{n=0}^{N-1} log(\frac{1}{\sigma \sqrt{2\pi}}\exp{[-1/2(\frac{x(n) - A cos (2\pi f_0 n + \phi)-\mu}{\sigma})^2]}
\end{equation}

\begin{equation}\label{key}
L(A,\phi) = \sum_{n=0}^{N-1} B + ({[-1/2(\frac{x(n) - A cos (2\pi f_0 n + \phi)-\mu}{\sigma})^2]}a
\end{equation}

\section{Show that the maximun likelihood estimators are the solution of the following equations}

\begin{equation}\label{key}
\frac{\partial L(A,\phi)}{\partial A} = \sum_{n=0}^{N-1} (\frac{x(n) - A cos (2\pi f_0 n + \phi)-\mu}{\sigma})(\frac{cos(2\pi f_0 n +\phi}{\sigma})
\end{equation}

\begin{equation}\label{key}
\frac{\partial L(A,\phi)}{\partial \phi} = \sum_{n=0}^{N-1} - (\frac{x(n) - A cos (2\pi f_0 n + \phi)-\mu}{\sigma})(\frac{Asin(2\pi f_0 n +\phi}{\sigma})
\end{equation}


\begin{equation}\label{key}
\frac{\partial L(A,\phi)}{\partial \phi} = 0 = \sum_{n=0}^{N-1} - (\frac{Ax(n)sin(2\pi f_0 n +\phi)}{\sigma^2}-\frac{A^2cos(2\pi f_0 n +\phi)sin(2\pi f_0 n +\phi)}{\sigma^2}) = 0
\end{equation}


\begin{equation}\label{key}
\frac{\partial L(A,\phi)}{\partial \phi} = 0 = \sum_{n=0}^{N-1} - (\frac{Ax(n)sin(2\pi f_0 n +\phi)}{\sigma^2}-\frac{A^2cos(2\pi f_0 n +\pi)sin(2\pi f_0 n +\phi)}{\sigma^2}) = 0
\end{equation}

\begin{equation}\label{key}
\frac{\partial L(A,\phi)}{\partial \phi} = 0 = \frac{A}{\sigma^2} \sum_{n=0}^{N-1} - (\frac{Ax(n)sin(2\pi f_0 n +\phi)-A^2cos(2\phi f_0 n +\pi)sin(2\phi f_0 n +\phi)}{\sigma^2}) = 0
\end{equation}


\begin{equation}\label{key}
\frac{\partial L(A,\phi)}{\partial \phi} = 0 = \sum_{n=0}^{N-1}(x(n)sin(2\pi f_0 n +\phi) - Acos(2\pi f_0 n +\pi))sin(2\phi f_0 n +\phi)) = 0
\end{equation}


\begin{equation}\label{key}
\frac{\partial L(A,\phi)}{\partial \phi} = 0 = \sum_{n=0}^{N-1}(x(n)sin(2\pi f_0 n +\phi) - A\sum_{n=0}^{N-1}cos(2\pi f_0 n +\phi)sin(2\pi f_0 n +\phi)) = 0
\end{equation}

Using the entity 

\begin{equation}\label{entitycossin}
\sum_{n=0}^{N-1}cos(2\pi f_0 n +\phi)sin(2\pi f_0 n +\phi)) \approx 0
\end{equation}

\begin{equation}\label{key}
\frac{\partial L(A,\phi)}{\partial \phi} = 0 = \sum_{n=0}^{N-1}(x(n)sin(2\pi f_0 n +\phi) = 0
\end{equation}

Developing the sum of angles

%\begin{equation}\label{key}
\begin{gather*} 
\frac{\partial L(A,\phi)}{\partial \phi} = 0 = \sum_{n=0}^{N-1}x(n)(sin(2\pi f_0 n +\phi)cos(2\pi f_0 n +\phi) + cos(2\pi f_0 n +\phi)sin(2\pi f_0 n +\phi)) = 0
\end{gather*}
%\end{equation}

\begin{equation}\label{key}
\frac{\partial L(A,\phi)}{\partial \phi} = 0 = \sum_{n=0}^{N-1}x(n)(sin(2\pi f_0 n +\phi)cos(\phi) = - \sum_{n=0}^{N-1}x(n)cos(2\pi f_0 n +\phi)sin(\phi)) 
\end{equation}



\begin{equation}\label{key}
\frac{\sum_{n=0}^{N-1}x(n)sin(2\pi f_0 n)}{\sum_{n=0}^{N-1}x(n)cos(2\pi f_0 n )} = \frac{sin(\phi )}{cos(\phi)} = tg(\phi)
\end{equation}


\begin{equation}\label{key}
\hat{\Phi}_{ML} = arctg(-\frac{\sum_{n=0}^{N-1}x(n)sin(2\pi f_0 n)}{\sum_{n=0}^{N-1}x(n)cos(2\pi f_0 n )})
\end{equation}


Making the derivation to the other parameter


\begin{gather*}\label{key}
\frac{\partial^2 L(A,\phi)}{\partial \phi^2} = \sum_{n=0}^{N-1} - (\frac{Asin(2\pi f_0 n +\phi)}{\sigma}) (\frac{Asin(2\pi f_0 n +\phi)}{\sigma})+\\
\sum_{n=0}^{N-1} - (\frac{x(n)-Acos(2\pi f_0 n +\phi)}{\sigma}) (\frac{Acos(2\pi f_0 n +\phi)}{\sigma})
\end{gather*}


\begin{gather*}\label{key}
\frac{\partial^2 L(A,\phi)}{\partial \phi^2} = \sum_{n=0}^{N-1} - (\frac{A^2sin^2(2\pi f_0 n +\phi)}{\sigma^2})+\\
\sum_{n=0}^{N-1} - (\frac{x(n)Acos(2\pi f_0 n +\phi)-A^2cos^2(2\pi f_0 n +\phi)}{\sigma^2})
\end{gather*}


\begin{gather*}\label{key}
\frac{\partial^2 L(A,\phi)}{\partial \phi^2} = \sum_{n=0}^{N-1} - (\frac{A^2sin^2(2\pi f_0 n +\phi)}{\sigma^2})+\\
\sum_{n=0}^{N-1} (\frac{w(n)}{\sigma^2})
\end{gather*}

\section{To compute the term of Fisher Matrix}

\begin{equation}\label{key}
-\sum_{n=0}^{N-1} E[\frac{\partial^2 L(A,\phi)}{\partial \phi^2}] = \sum_{n=0}^{N-1}\frac{A^2sin^2(2\pi f_0 n +\phi)}{\sigma^2}
\end{equation}


\begin{equation}\label{key}
-\sum_{n=0}^{N-1} E[\frac{\partial^2 L(A,\phi)}{\partial \phi^2}] = \frac{A^2N}{\sigma^2 2}
\end{equation} 


\begin{gather*}\label{key}
[\frac{\partial^2 L(A,\phi)}{\partial \phi^2}] = -\sum_{n=0}^{N-1} \frac{- cos(2\pi f_0 n +\phi)}{\sigma} \frac{Asin(2\pi f_0 n +\phi)}{\sigma} \\  + \sum_{n=0}^{N-1} - \frac{x(n)- Acos(2\pi f_0 n +\phi)}{\sigma} \frac{sin(2\pi f_0 n +\phi)}{\sigma}
\end{gather*}

\begin{equation}\label{key}
-\sum_{n=0}^{N-1} E[\frac{\partial^2 L(A,\phi)}{\partial \phi^2}] = -\sum_{n=0}^{N-1} \frac{- cos(2\pi f_0 n +\phi)}{\sigma} \frac{Asin(2\pi f_0 n +\phi)}{\sigma}
\end{equation} 

Using the simplification in equation \ref{entitycossin}

\begin{equation}\label{key}
-\sum_{n=0}^{N-1} E[\frac{\partial^2 L(A,\phi)}{\partial \phi^2}] = 0
\end{equation} 

\section{Verify thath thw maximun-likelihood estimator of the amplitude is unbiase if $\phi = \hat{\Phi}_{ML}$}

Condition pour $\hat{\Phi}_{ML} = \Phi$

\begin{equation}\label{key}
tan(\hat{\Phi}_{ML} )= tan(\Phi)
\end{equation}



\begin{gather*}
\frac{\sum_{n=0}^{N-1}x(n)sin(2\pi f_0n+\phi)}{\sum_{n=0}^{N-1}x(n)cos(2\pi f_0n+\phi)} = \\
\frac{\sum_{n=0}^{N-1}Acos(2\pi f_0n+\phi)+w(n)sin(2\pi f_0n+\phi)}{\sum_{n=0}^{N-1}Acos(2\pi f_0n+\phi)+w(n)cos(2\pi f_0n+\phi)}
\end{gather*}


\begin{gather*}
\frac{\sum_{n=0}^{N-1}x(n)sin(2\pi f_0n+\phi)}{\sum_{n=0}^{N-1}x(n)cos(2\pi f_0n+\phi)} = \\
\frac{\sum_{n=0}^{N-1}Acos(2\pi f_0n+\phi)+w(n)sin(2\pi f_0n+\phi)}{\sum_{n=0}^{N-1}Acos(2\pi f_0n+\phi)+w(n)cos(2\pi f_0n+\phi)}
\end{gather*}



\begin{gather*}
\frac{AN/2 sin(\phi)\sum_{n=0}^{N-1}w(n)sin(2\pi f_0n)}{AN/2 sin(\phi)\sum_{n=0}^{N-1}w(n)cos(2\pi f_0n)} \approx tan(\Phi)
\end{gather*}

This term will be next to $tan(\Phi)$ when 

\begin{equation}\label{key}
\sum_{n=0}^{N-1}|w(n)sin(2\pi f_0n)|<<|AN/2 sin(\phi)|
\end{equation}

and when we have 
\begin{equation}\label{key}
\sum_{n=0}^{N-1}|w(n)cos(2\pi f_0n)|<<|AN/2 cos(\phi)|
\end{equation}

The noise have a normal distribution so the sum of noise have the next distribution

\begin{equation}\label{key}
\sum_{n=0}^{N-1}|w(n)sin(2\pi f_0n)|\sim \textit{N}(0,N\sigma^2/2)
\end{equation}


We take a probabilité of 3$\sigma$
\begin{gather*}\label{key}
3\sqrt{\sigma^2 N/2} = 3\sqrt{N/2}\sigma << AN/2 sin(\phi)\\
3 << \frac{A\sqrt{N}}{\sigma}sin(\phi)
\end{gather*}

For larges values of N $ \hat{\Phi}_ML \approx \phi $

We cans observe that the parameters A, $ \sigma $ and N can modified
the estimator properties.

The ratio $ \frac{A^2}{\sigma^2} $ represent the RSB %SEARCH THIS DEFINITION

\section{Give the condition under which $ \hat{\Phi}_ML \approx \phi $ different from $ E|\hat{\Phi}_ML| \approx \phi $ }

Using

\begin{gather*}
sen(\alpha)cos(\alpha) = 1/2 sin(2\alpha)\\
cos^2(\alpha) = \frac{1+cos(2\alpha)}{2}\\
sin^2(\alpha) = \frac{1-cos(2\alpha)}{2}
\end{gather*}

In the entities of the copy, we obtain 

\begin{equation}\label{key}
\sum_{n=0}^{N-1}cos(4\pi f_0 n) \approx 0
\end{equation}

\begin{equation}\label{key}
\sum_{n=0}^{N-1}sin(4\pi f_0 n) \approx 0
\end{equation}

And the last entity 

\begin{equation}\label{key}
\sum_{n=0}^{N-1}(e^{j4\pi f_0n}) \approx 0
\end{equation}


$ f_0t(0,1/2) $ ne sont pas des valeurs a prendre a cause du theoreme de sannon

Donc il faut pas que fo = 1/2 et 0

Il faut excluire les axes
$ \phi = 4\pi fo  $

$ \phi = 2\pi $

%------------------------------------------------------------
%	SECTION 2
%------------------------------------------------------------
\chapter{Implémentation en MATLAB{\small \circledR}}




\section{Estimation de l'information mutuelle}



%----------------------------------------------------------------------------------------
%	SECTION 3
%----------------------------------------------------------------------------------------


\section{Conclusion}


%----------------------------------------------------------------------------------------
%	FIN
%----------------------------------------------------------------------------------------



\end{document}























